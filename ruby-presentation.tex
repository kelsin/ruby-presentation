\usepackage{color}
\usepackage{float}
\usepackage{minted}

\input xy
\xyoption{all}

\usepackage{beamerthemesplit}
\usetheme{Darmstadt}
\usecolortheme{crane}

\setbeamercovered{dynamic}

\title{Ruby: Why We Love It}
\subtitle{https://github.com/Kelsin/ruby-presentation}
\author{Christopher Giroir}
\date{October 4th, 2011}

% Commands
\newminted{ruby}{gobble=4,linenos,frame=leftline}
\newminted{scheme}{gobble=4,linenos,frame=leftline}
\newminted[console]{bash}{gobble=4,linenos,frame=leftline}
\usemintedstyle{tango}

\begin{document}

\maketitle

\begin{frame}
  \titlepage
\end{frame}

\begin{frame}
  \tableofcontents
\end{frame}

\section{The Language}

\subsection{Basics}

\begin{frame}
  \frametitle{Overview}
  \begin{itemize}
  \item Inspired by Smalltalk (which I love)
  \pause
  \item Also draws from Perl, Eiffel, Ada and LISP
  \pause
  \item Includes a REPL
  \pause
  \item Built for developers as a language they would love to use
  \pause
  \item Dynamic, strict, reflective, object oriented
  \pause
  \item Everything is an expression (even statements)
  \pause
  \item Everything is executed imperatively (even declarations)
  \end{itemize}
\end{frame}

\begin{frame}
  \frametitle{Object Oriented}
  \begin{itemize}
  \item Everything is an object
  \pause
  \item Single Inheritance
  \pause
  \item Modules can be mixed in
  \pause
  \item Dynamic Dispatch
  \end{itemize}
\end{frame}

\begin{frame}[fragile]
  \frametitle{Simple Code}
  \begin{rubycode}
    5.times { print "Hello" }
  \end{rubycode}
  \pause
  This outputs:
  \begin{console}
    Hello
    Hello
    Hello
    Hello
    Hello
    => 5
  \end{console}
\end{frame}

\begin{frame}[fragile]
  \frametitle{Types}
  \begin{rubycode}
    # Strings
    s = 'Testing'

    # Interpreted Strings
    t = "Double #{str}"

    # Keywords
    k = :chris

    # Arrays
    a = [1,2,3]

    # Hashes
    h = { :key => 'value', :chris => 'awesome' }
  \end{rubycode}
\end{frame}

\begin{frame}[fragile]
  \frametitle{Classes}
  \begin{rubycode}
    class Box
      def initialize(w,h,d)
        @width = w
        @height = h
        @depth = d
      end

      def volume
        @width * @height * @depth
      end
    end

    box = Box.new(2,2,2)
    box.volume # => 8
  \end{rubycode}
\end{frame}

\begin{frame}[fragile]
  \frametitle{Simple Inheritance}
  \begin{rubycode}
    class JackInTheBox < Box
      def initialize(msg)
        @msg = msg
        super(3,3,3)
      end

      def open
        puts @msg
      end
    end

    jbox = JackInTheBox.new('Surprise!')
    jbox.volume # => 27
    jbox.open # prints 'Surprise!'
  \end{rubycode}
\end{frame}

\begin{frame}[fragile]
  \frametitle{Control}
  \begin{rubycode}
    while true == false
      if var == 5
        break
      end

      begin
        var - 1
      end while var < 4

      next if var == 6
    end
  \end{rubycode}
\end{frame}

\begin{frame}[fragile]
  \frametitle{Blocks}
  \begin{rubycode}
    [1,2,3].each { |n| puts n }
  \end{rubycode}
  \pause
  This outputs:
  \begin{console}
    1
    2
    3
    => [1,2,3]
  \end{console}
\end{frame}

\begin{frame}[fragile]
  \frametitle{Block Syntax}
  \begin{rubycode}
    5.upto(10) { |n| puts n }
  \end{rubycode}

  This is exactly the same as the following:

  \begin{rubycode}
    5.upto(10) do |n|
      puts n
    end
  \end{rubycode}
\end{frame}

\subsection{Why We Love It}

\begin{frame}[fragile]
  \frametitle{Attribute Methods}
  \begin{rubycode}
    class Person
      def name
        @name
      end
      def social=(s)
        @social = s
      end
      def age
        @age
      end
      def age=(a)
        @age = a
      end
    end
  \end{rubycode}
\end{frame}

\begin{frame}[fragile]
  \frametitle{The Easy Way}
  \begin{rubycode}
    class Person
      attr_reader :name
      attr_writer :social
      attr_accessor :age
    end
  \end{rubycode}
\end{frame}

\begin{frame}[fragile]
  \frametitle{Why Blocks}
  \begin{schemecode}
    (map (lambda (n)
           (+ n 5))
         '(1 2 3))
  \end{schemecode}
  Becomes:
  \begin{rubycode}
    [1,2,3].map do |n|
      n + 5
    end
  \end{rubycode}
  Results in:
  \begin{console}
    => [6,7,8]
  \end{console}
\end{frame}

\subsection{Gems}

\begin{frame}[fragile]
  \frametitle{Modules}
  \begin{rubycode}
    module Voice
      def say(msg)
        puts msg
      end
    end

    class Person
      include Voice
    end

    p = Person.new
    p.say('Hello') # prints 'Hello'
  \end{rubycode}
\end{frame}

\begin{frame}[fragile]
  \frametitle{Using Gems}
  Require loads in files
  \begin{rubycode}
    require 'saver' # pulls in 'saver.rb'
  \end{rubycode}
  Gems allow us to not deal with paths
  \begin{rubycode}
    require 'rubygems'
    require 'saver'

    class Item
      include Saver
    end
  \end{rubycode}
\end{frame}

\begin{frame}[fragile]
  \frametitle{Writing Gems}
  \begin{rubycode}
    Gem::Specification.new do |s|
      s.name        = "saver"
      s.version     = Saver::VERSION
      s.authors     = ["Christopher Giroir"]
      s.email       = ["kelsin@valefor.com"]
      s.homepage    = "http://kelsin.github.com/saver/"

      s.files         = `git ls-files`.split("\n")
      s.require_paths = ["lib"]

      s.add_dependency 'activesupport', '~> 3.0.0'
      s.add_dependency 'mongo_mapper'
    end
  \end{rubycode}
\end{frame}

\section{Tools}

\subsection{Bundler}

\begin{frame}
  \frametitle{Why Bundler?}
  \begin{itemize}
  \item Many projects (i.e. rails apps) are not gems themselves
  \item They do have gem dependencies
  \pause
  \item Easy way to install and keep track of these dependencies
  \pause
  \item Making sure ONLY the proper gems are used
  \end{itemize}
\end{frame}

\begin{frame}[fragile]
  \frametitle{The Gemfile}
  \begin{rubycode}
    source 'http://tools1.savewave.com/rubygems'
    source 'http://rubygems.org'

    gem 'rails', '3.0.7'

    gem 'sw-model', '0.13.0'

    group :development, :test do
      gem "rspec"
    end
  \end{rubycode}
\end{frame}

\begin{frame}[fragile]
  \frametitle{Using Bundler}
  \begin{console}
    # Install the gems from the Gemfile
    bundle install

    # Update gems to new versions
    bundle update

    # Execute command with proper gems
    bundle exec rake spec
  \end{console}
  In your ruby code
  \begin{rubycode}
    require "rubygems"
    require "bundler/setup"
    require "saver"
  \end{rubycode}
\end{frame}

\begin{frame}[fragile]
  \frametitle{Gemfile.lock}
  \begin{itemize}
  \item When you initially install versions are saved to Gemfile.lock
  \pause
  \item After they are only updated on bundle update
  \pause
  \item SHOULD be checked into version control
  \item Protects from version updates
  \end{itemize}
\end{frame}

\subsection{RVM}

\begin{frame}[fragile]
  \frametitle{Why RVM?}
  \begin{itemize}
  \item Different projects might use different versions of rails
  \pause
  \item Different projects might use different ruby interpreters
    \begin{itemize}
    \item Ruby
    \item JRuby
    \item Rubinus
    \end{itemize}
  \pause
  \item While bundler helps, complete gem isolation is better!
  \pause
  \item It's nice to keep your system ruby separate and not update it
  \end{itemize}
\end{frame}

\begin{frame}[fragile]
  \frametitle{Using RVM}
  \begin{console}
    # Install the default 1.9.2 ruby interpretor
    rvm install 1.9.2

    # Switch to using 1.9.2
    rvm use 1.9.2

    # List installed rubies
    rvm list
  \end{console}
\end{frame}

\begin{frame}[fragile]
  \frametitle{RVM Gemsets}
  \begin{console}
    # Create a new gemset
    rvm gemset create savingstar-web

    # List gemsets
    rvm gemset list

    # Switch to a ruby and gemset together
    rvm use 1.9.2@savingstar-web
  \end{console}
\end{frame}

\begin{frame}[fragile]
  \frametitle{.rvmrc}
  \begin{itemize}
  \item A .rvmrc file per project allows you to say which ruby and gemset to use
  \pause
  \item Should be in source control. Helps RVM users out, ignored for others
  \pause
  \item It's a shell script that's executed everytime you cd (very unsafe)
  \pause
  \item Makes life very easy however!
  \end{itemize}
  \pause
  \begin{rubycode}
    rvm use 1.9.2@saveingstar-web --create
  \end{rubycode}
\end{frame}

\end{document}
